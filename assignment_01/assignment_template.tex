% Enable warnings about problematic code
\RequirePackage[l2tabu, orthodox]{nag}

\documentclass{resources/WeSTassignment}
\usepackage{tabularx}
\usepackage{booktabs}
\usepackage[utf8]{inputenc}
\usepackage{amsmath}
\usepackage{graphics}
\usepackage{graphicx}
\usepackage{changebar}
\usepackage{latexsym}
\usepackage{stmaryrd}
\usepackage{booktabs}
\usepackage{amsmath}
\usepackage{wasysym}
\usepackage[export]{adjustbox}
\usepackage[thinlines]{easytable}
\usepackage{framed}
\usepackage{color}
\usepackage{footnote}
\usepackage{listings}
\usepackage{framed}
\usepackage{tikz}
%\usepackage[table]{xcolor}

% The lecture title, e.g. "Web Information Retrieval".
\lecture{Machine Learning and Data Mining}
% The names of the lecturer and the instructor(s)
\author{%
  Raphael Menges\\{\normalsize\mailto{raphaelmenges@uni-koblenz.de}} 
}
% Assignment number.
\assignmentnumber{1}
% Institute of lecture.
\institute{%
  Institute of Web Science and Technologies\\%
  Department of Computer Science\\%
  University of Koblenz-Landau%
}
% Date until students should submit their solutions.
\datesubmission{16.11.2020, CEST 23:59}
% Date on which the assignments will be discussed in the tutorial.

% Specify bib file location.
\addbibresource{bibliography.bib}

\begin{document}

\maketitle
The exercises in this assignment are of theoretical nature and may not be solved by
execution of high-level Python commands but through manual step-by-step calculations which must be included in submissions. For this assignment it is also allowed
to upload a single .pdf file generated from LATEX code or scanned and compressed (!)
handwritten solutions. \\
\section{Statistics\hfill{18 points}}
\subsection{\hfill{6 points}}
 

\subsection{\hfill{6 points}}


\section{Error Calculation \hfill{12 points}}
You are given many computed outputs y\textsubscript{i} and desired outcomes \^y\textsubscript{i}.Provide the following error measures in regards to y\textsubscript{i} and \^y\textsubscript{i} by writing down their formula and a short description about their characteristics, i.e., the behavior in regard to the difference between computed and desired outcome.
\subsection{Sum of Square Error (SSE)}
\[ SSE = \sum_{n=1}^{n} (\hat{y_i} - y_i)^2 \]
y\textsubscript{i} : computed output \\
\^y\textsubscript{i} : desired output

\begin{description}
\item[$\bullet$]The error will be the summation of differences squared between y\textsubscript{i} (computed value) and \^y\textsubscript{i} (actual value).
\item[$\bullet$] The difference is squared in order to avoid positive terms cancelling out the negative terms
  
\end{description}


    

\section{Research tasks \hfill {20 points}}
.
\subsection{Collision}


\subsection{IPv6}


\section{Routing Table \hfill {20 points}}










\end{document}